\documentclass[aspectratio=169]{beamer}
 
\usepackage[utf8]{inputenc}
\usepackage{lipsum}

%%%%%%%%%%%%%%%%%%%%%%%%%%%%%%%%%%%%%%%%%%%%%%%%
% Set Helvetica Font in Text and Math in LaTeX %
%%%%%%%%%%%%%%%%%%%%%%%%%%%%%%%%%%%%%%%%%%%%%%%%
\usepackage[helvet]{sfmath}
\usepackage[scaled=1]{helvet}
\usepackage[T1]{fontenc}
\renewcommand\familydefault{\sfdefault} 
\everymath={\sf}
\usepackage{fontspec}
\usepackage{amsmath,amssymb,amsfonts}
\usepackage{subcaption}

\hypersetup{
    colorlinks=true,
    linkcolor=blue,
    filecolor=magenta,      
    urlcolor=cyan,
    pdftitle={Overleaf Example},
%    pdfpagemode=FullScreen,
}

% use settings files
\input{configuration_files/beamer_theme.tex}
\input{configuration_files/quotations.tex}

% Other packages
\usepackage{enumitem}
\setitemize{label=\usebeamerfont*{itemize item}%
  \usebeamercolor[fg]{itemize item}
  \usebeamertemplate{itemize item}}

\usepackage{amssymb}
\newcommand*{\QEDA}{\hfill\ensuremath{\blacksquare}}

\usepackage{datetime}
\newdateformat{specialdate}{\twodigit{\THEDAY} \ \monthname[\THEMONTH] \THEYEAR}
%%%%%%%%%%%%%%%%%%%%%%%%%%%%%%%%%%%%%%%%%%%%%%%%%%%%%%%%%%%%%%%%%

\newcommand{\nologo}{\setbeamertemplate{logo}{}}
\newcommand{\yeslogo}{\logo{\vspace{-0.5cm}\includegraphics[width=1.5cm]{./images/HR_logo_hringur_lowres}\hspace{0.5cm}}}

\yeslogo

\newif\ifpause
\pausetrue
%\pausefalse
\newcommand{\mypause}{\ifpause \pause \fi}

%Information to be included in the title page:
\title{Evaluation of Orientation Ambiguity\\and Detection Rate in April Tag and WhyCode}
\author{Joshua Springer, Marcel Kyas}
\institute{Reykjavik University\\Department of Computer Science}

\date{\specialdate\today}

\begin{document}

\maketitle

\nologo

\begin{frame}{Fiducial Markers}
	\vspace*{\fill}
	\begin{figure}[]
	    \centering
	    \begin{subfigure}[b]{0.23\linewidth}
		\includegraphics[width=\textwidth]{./images/tagCustom48h12_00002_00001_00000}
		    \caption{April Tag 48h12~\cite{apriltag3_paper}}
		\label{figure:apriltag48h12}
	    \end{subfigure}
		\hspace{0.01\linewidth}
	    \begin{subfigure}[b]{0.23\linewidth}
		\includegraphics[width=\textwidth]{./images/tagCustom24h10_00002_00001_00000}
		    \caption{April Tag 24h10~\cite{fiducial_precursor_evaluation}}
		\label{figure:apriltag24h10}
	    \end{subfigure}
		\hspace{0.01\linewidth}
	    \begin{subfigure}[b]{0.23\linewidth}
		\includegraphics[width=\textwidth]{./images/whycode_20_8}
		    \caption{WhyCode (Orig)\cite{whycode_paper}}
		\label{figure:whycode_single}
	    \end{subfigure}
		\hspace{0.01\linewidth}
	    \begin{subfigure}[b]{0.23\linewidth}
		\includegraphics[width=\textwidth]{./images/whycode_multi}
		    \caption{WhyCode Multi~\cite{fiducial_precursor_evaluation}}
		\label{figure:whycode_bundle}
	    \end{subfigure}
	    \label{figure:marker_setup}
	\end{figure}
	\mypause
%	\vspace{1cm}
	\begin{itemize}
		\item Marker \emph{position} $\rightarrow$ accurate
		\item Marker \emph{orientation} $\rightarrow$ ambiguous
		\item Application: autonomous precision drone landing
	\end{itemize}
	\vspace*{\fill}
\end{frame}

\begin{frame}{WhyCode (Orig)}
	\begin{columns}
	\begin{column}{0.4\textwidth}
		\centering
		\includegraphics[width=\linewidth]{./images/whycode_orig_both_solutions_cropped.png}
	\end{column}
	\begin{column}{0.6\textwidth}
	\begin{itemize}
		\item Position $\rightarrow$ camera intrinsics, marker size (known a priori)\mypause
		\item Orientation $\rightarrow$ two candidates (yellow/green ellipses)
		\begin{itemize}
			\item Which one fits better?
			\item Minimize variance of intersection length of sampling circle with each tooth
		\end{itemize}
	\end{itemize}
	\end{column}
	\end{columns}
\end{frame}

\begin{frame}{WhyCode ``Ellipse''}
	\begin{columns}
	\begin{column}{0.4\textwidth}
		\centering
		\includegraphics[width=\linewidth]{./images/whycode_ellipse_both_solutions_cropped.png}
	\end{column}
	\begin{column}{0.6\textwidth}
	\begin{itemize}
		\item Changes the method for choosing the orientation candidate
		\begin{itemize}
			\item Original sampling predicts tooth transition locations
			\item Sample again along radii through tooth centers
			\item Try to center radial sampling lines on white-to-black transitions
		\end{itemize}
	\end{itemize}
	\end{column}
	\end{columns}
\end{frame}

\begin{frame}{WhyCode ``Multi''}
	\begin{columns}
	\begin{column}{0.4\textwidth}
		\centering
		\includegraphics[width=\linewidth]{./images/cropped_whycode_3_8_jiri_example.png}
	\end{column}
	\begin{column}{0.6\textwidth}
	\begin{itemize}
		\item Assumes coplanar marker arrangements
		\item Samples as in WhyCode Orig
		\item Uses only marker size, positions and IDs.
		\item Regresses a plane to the marker positions
		\item Assigns the plane's orientation to all of the markers
	\end{itemize}
	\end{column}
	\end{columns}
\end{frame}

\begin{frame}{April Tag 48h12}
	\begin{columns}
	\begin{column}{0.4\textwidth}
		\centering
		\includegraphics[width=\linewidth]{./images/tagCustom48h12_00002_00001_00000.png}
	\end{column}
	\begin{column}{0.6\textwidth}
	\begin{itemize}
		\item 36 outer ID bits + 12 inner ID bits
		\item 28 black border bits
		\item 20 white border bits
		\item 4 undefined, center bits (for marker embedding)
		\item 42,211 markers in a single hash table in RAM
	\end{itemize}
	\end{column}
	\end{columns}
\end{frame}

\begin{frame}{April Tag 24h10}
	\begin{columns}
	\begin{column}{0.4\textwidth}
		\centering
		\includegraphics[width=\linewidth]{./images/tagCustom24h10_00002_00001_00000.png}
	\end{column}
	\begin{column}{0.6\textwidth}
	\begin{itemize}
		\item 24 outer ID bits
		\item 16 black border bits
		\item 8 white border bits
		\item 1 undefined center bit (for marker embedding)
		\item 18 total markers
	\end{itemize}
	\end{column}
	\end{columns}
\end{frame}

\begin{frame}{ROS Message Attributes}
	\begin{itemize}
		\item For autonomous landing application
		\item A \textbf{position target} $P$ in the a relative
			``east, north, up'' (ENU) coordinate frame
		\item The marker's \textbf{orientation quaternion} $q$
		\item The \textbf{normalized pixel position} $u_n, v_n \in [-1,1]$
			of the center of each marker
		\item The marker's \textbf{orientation components}: yaw, pitch, and roll
			(where not originally exposed)
	\end{itemize}
\end{frame}

\begin{frame}{Quantifying Orientation Ambiguity}
	\centering
	\renewcommand{\arraystretch}{2}
	\begin{tabular}{ll}
		\textbf{Linear Discontinuities} & \textbf{Angular Discontinuities}\\
		$P = \langle p_e, p_n, p_u \rangle$ & $q = \langle q_w, q_x, q_y, q_z \rangle$\\
		$\frac{p_{x,i+1}}{p_{x,i}} < \theta_l < 0$ & $s_a = \frac{\mathrm{dist} \left( q_i, q_{i+1} \right) }{\Delta t} > \theta_a > 0$\\
		$\theta_l = -0.8$ & $\theta_a = 1.0 \frac{\mathrm{rad}}{s}$
	\end{tabular}
	\begin{itemize}
		\item Single marker arrangement showing all markers
		\item All markers fully contained in every frame
		\item Discontinuity occurs when both conditions are true
		\item Discontinuity rate: $r_d = \frac{d}{n}$: $d$ is number of discontinuities, $n$ is number of detections
	\end{itemize}
\end{frame}

\begin{frame}{Quantifying Detection Rate}
	\begin{itemize}
		\item One static marker per video
		\item Multiple distances
		\item Constant camera, framerate, resolution
		\item Detection rate: $F=\frac{n}{t}$: $n$ is number of detections, $t$ is the length in seconds of each test case
		\item Raspberry Pi 4 with 2 GB of RAM
	\end{itemize}
\end{frame}

\begin{frame}{Example Test Case}
	\centering
	\includegraphics[width=0.75\linewidth]{images/orbit-left-right-3_apriltag48h12_position-target}
%	\includegraphics[width=0.49\linewidth]{images/orbit-left-right-4_apriltag48h12_position-target}
%	\includegraphics[width=0.95\linewidth]{images/orbit-left-right-4_apriltag48h12_angular-velocity}
\end{frame}

\begin{frame}{Example Test Case with Discontinuities}
	\centering
%	\includegraphics[width=0.49\linewidth]{images/orbit-left-right-3_apriltag48h12_position-target}
	\includegraphics[width=0.75\linewidth]{images/orbit-left-right-4_apriltag48h12_position-target}
%	\includegraphics[width=0.95\linewidth]{images/orbit-left-right-4_apriltag48h12_angular-velocity}
\end{frame}

\begin{frame}{Example Test Case - Angular Speed}
	\centering
%	\includegraphics[width=0.49\linewidth]{images/orbit-left-right-3_apriltag48h12_position-target}
%	\includegraphics[width=0.49\linewidth]{images/orbit-left-right-4_apriltag48h12_position-target}
	\includegraphics[width=0.7\linewidth]{images/orbit-left-right-4_apriltag48h12_angular-velocity}
\end{frame}

\begin{frame}{Results: Discontinuity Rates}
    \centering
    \includegraphics[width=0.6\linewidth]{./images/violin_plot_five_member}
%    \includegraphics[width=0.49\linewidth]{./images/violin_plot_speed_five_member}
\end{frame}

\begin{frame}{Results: Detection Rates}
    \centering
%    \includegraphics[width=0.49\linewidth]{./images/violin_plot_five_member}
    \includegraphics[width=0.6\linewidth]{./images/violin_plot_speed_five_member}
\end{frame}

\begin{frame}{Main Messages}
	\begin{itemize}
		\item Fiducial markers: orientation is hard to determine
		\begin{itemize}
			\item Orientation ambiguity manifests as sign flips
			\item Present in all fiducial systems, but at different rates
		\end{itemize}
		\item Detection rate
		\begin{itemize}
			\item Limited by hardware capabilities
			\item Limited by software architecture
			\item Some faster than others
		\end{itemize}
	\end{itemize}
\end{frame}

\begin{frame}{References}
	\setbeamertemplate{bibliography item}[triangle]
	\bibliographystyle{plain}
	\bibliography{references}
\end{frame}

\end{document}
